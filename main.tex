\documentclass{article}

\usepackage[hidelinks]{hyperref}
\usepackage{url}
\usepackage{todonotes}
\usepackage{setspace}
\usepackage{tcolorbox}


\usepackage[letterpaper,top=2cm,bottom=2cm,left=2cm,right=2cm,marginparwidth=1.75cm]{geometry}


%\usepackage{charter}
%\usepackage{calc}

\usepackage{multirow}
% \usepackage[cp1250]{inputenc}
% \usepackage[T1]{fontenc}
% \usepackage{calligra}
\usepackage{layout}

% AMS
\usepackage{amsmath}
\usepackage{amsthm}
\usepackage{amssymb}
\usepackage{amsfonts}
\usepackage{mathtools}
\usepackage{siunitx}

\usepackage{enumitem}

\usepackage{bbm}

\usepackage{url}
\usepackage{color}
\usepackage{graphicx}
\usepackage{verbatim}
\usepackage{subcaption}

\usepackage{natbib}

\usepackage{apptools}
\usepackage{thmtools}
\usepackage{algorithm, algpseudocode}

\usepackage[nice]{nicefrac}

% paper specific macros
\newcommand{\g}{{\color{red}g}}
\newcommand{\h}{{\color{blue}h}}

% general macros

\newcommand{\norm}[1]{\left\| #1 \right\|}
\newcommand{\bnorm}[1]{\left\| #1 \right\|_{\mB}}
\newcommand{\sqnorm}[1]{\left\| #1 \right\|^2}
\newcommand{\lin}[1]{\left\langle #1\right\rangle} % inner product
\newcommand{\inp}[2]{\left\langle#1,#2\right\rangle} % inner product
\newcommand{\binp}[2]{\left\langle#1,#2\right\rangle_{\mB}}

\newcommand{\parens}[1]{\left( #1 \right)}
\newcommand{\brac}[1]{\left\{ #1 \right\}}


% caligraphic
\newcommand{\cA}{\mathcal{A}}
\newcommand{\cB}{\mathcal{B}}
\newcommand{\cC}{\mathcal{C}}
\newcommand{\cD}{\mathcal{D}}
\newcommand{\cF}{\mathcal{F}}
\newcommand{\cH}{\mathcal{H}}
\newcommand{\cL}{\mathcal{L}}
\newcommand{\cK}{\mathcal{K}}
\newcommand{\cJ}{\mathcal{J}}
\newcommand{\cN}{\mathcal{N}}
\newcommand{\cO}{\mathcal{O}}
\newcommand{\cQ}{\mathcal{Q}}
\newcommand{\cS}{\mathcal{S}}
\newcommand{\cT}{\mathcal{T}}
\newcommand{\cU}{\mathcal{U}}
\newcommand{\cW}{\mathcal{W}}
\newcommand{\cX}{\mathcal{X}}
\newcommand{\cY}{\mathcal{Y}}
\newcommand{\cZ}{\mathcal{Z}}

% bold matrices
\newcommand{\mA}{\mathbf{A}}
\newcommand{\mB}{\mathbf{B}}
\newcommand{\mC}{\mathbf{C}}
\newcommand{\mM}{\mathbf{M}}
\newcommand{\mN}{\mathbf{N}}
\newcommand{\mI}{\mathbf{I}}
\newcommand{\mJ}{\mathbf{J}}
\newcommand{\mL}{\mathbf{L}}
\newcommand{\mD}{\mathbf{D}}
\newcommand{\mO}{\mathbf{O}}
\newcommand{\mH}{\mathbf{H}}
\newcommand{\mP}{\mathbf{P}}
\newcommand{\mS}{\mathbf{S}}
\newcommand{\mZ}{\mathbf{Z}}
\newcommand{\mU}{\mathbf{U}}
\newcommand{\mQ}{\mathbf{Q}}
\newcommand{\mV}{\mathbf{V}}
\newcommand{\mX}{\mathbf{X}}
\newcommand{\mW}{\mathbf{W}}

% bold vectors
\newcommand{\ma}{\mathbf{a}}
\newcommand{\mb}{\mathbf{b}}
\newcommand{\mc}{\mathbf{c}}
\newcommand{\md}{\mathbf{d}}
\newcommand{\mx}{\mathbf{x}}
\newcommand{\my}{\mathbf{y}}
\newcommand{\mz}{\mathbf{z}}
\newcommand{\mw}{\mathbf{w}}
\newcommand{\mzero}{\mathbf{0}}


\newcommand{\proj}{\textnormal{Proj}^{\perp}}
\newcommand{\trbar}{\overline{\textnormal{Tr}}}

\newcommand{\prox}{\textnormal{prox}}
\newcommand{\teqp}{t_\textnormal{prox}}
\newcommand{\Teq}{t^*}
\newcommand{\tprox}{t^{\prox}}

\newcommand{\flr}[1]{\left\lfloor #1\right\rfloor} 
\newcommand{\ceil}[1]{\left\lceil #1\right\rceil} 

\newcommand{\hf}{\hat{f}}
\newcommand{\bprox}{\textnormal{brox}_f^t}

% Some duplicate notations on prox and the envelope function
\newcommand{\rbrac}[1]{\left(#1\right)}
\newcommand{\sbrac}[1]{\left[#1\right]}
\newcommand{\cbrac}[1]{\left\{#1\right\}}
\newcommand{\BProx}{\textnormal{brox}}
\newcommand{\BProxSub}[3]{\textnormal{{brox}}^{#1}_{#2}\left(#3\right)}
\newcommand{\ProxSub}[2]{\textnormal{{prox}_{#1}}\left(#2\right)}
\newcommand{\MoreauSub}[3]{{M}^{#1}_{#2}\rbrac{#3}}
\newcommand{\BMoreauSub}[3]{{N}^{#1}_{#2}\rbrac{#3}}
\newcommand{\BMoreau}{{N}}
\newcommand{\inner}[2]{\left\langle #1, #2 \right\rangle}
\newcommand{\dist}[2]{\textnormal{dist}\rbrac{#1, #2}}
\newcommand{\breg}[3]{D_{#1}\rbrac{#2, #3}}
\newcommand{\cmin}[1]{c^{\min}_t\rbrac{x_{#1}}}
\newcommand{\cmax}[1]{c^{\max}_t\rbrac{x_{#1}}}
\newcommand{\BpproxSub}[3]{\textnormal{{bprox}}^{#1}_{#2}\left(#3\right)}
\newcommand{\Bpprox}{\textnormal{bprox}}
\newcommand{\Sprox}{\textnormal{{Sprox}}}
\newcommand{\SproxSub}[4]{\textnormal{{Sprox}}^{#1}_{#2, #3}\rbrac{#4}}


\newcommand{\supp}{\textnormal{supp}}
\newcommand{\prog}{\textnormal{prog}}

\newcommand{\grad}{\nabla}
\newcommand{\sgrad}{\widehat{\nabla}}

% strange stuff
\newcommand{\hotidea}{{\color{red}\bf HOT IDEA: }}
\newcommand{\done}{{\color{blue}\bf DONE: }}
\newcommand{\del}[1]{}
\let\la=\langle
\let\ra=\rangle

% basic macros
\newcommand{\R}{\mathbb{R}} % reals
\newcommand{\Z}{\mathbb{Z}}
\newcommand{\N}{\mathbb{N}}
\newcommand{\I}{\mathbb{I}}
\newcommand{\U}{\mathbb{U}}
\newcommand{\PermComp}{\U\mathbb{P}}

\newcommand{\st}{\;:\;} % such that
\newcommand{\eqdef}{\stackrel{\text{def}}{=}}
% \newcommand{\eqdef}{:=} %\vcentcolon


\newcommand{\clip}[1]{{\rm {\color{mydarkred}clip}_{\tau}}\left(#1\right)}


\newcommand{\nnz}[1]{{\color{red}\|#1\|_0}}

% sets
\DeclareMathOperator{\card}{card}         % cardinality of a set
\DeclareMathOperator{\diam}{diam}        % diameter of a set
\DeclareMathOperator{\vol}{vol}               % volume of a set


\newcommand{\Prob}[1]{{\mathbf{Prob}}\left(#1\right)} % probability
\newcommand{\ProbCond}[2]{\mathbf{Prob}\left(\left.#1\right\vert#2\right)}

% statistics
\newcommand{\Exp}[1]{{\mathbb{E}}\left[#1\right]}
\newcommand{\Expxi}[1]{{\mathbb{E}_{\xi}}\left[#1\right]}
\newcommand{\ExpCond}[2]{{\mathbb{E}}\left[\left.#1\right\vert#2\right]}
\newcommand{\ExpSub}[2]{{\mathbb{E}}_{#1}\left[#2\right]}
\newcommand{\ExpSubCond}[3]{{\mathbb{E}}_{#1}\left[#2\vert#3\right]}
\DeclareMathOperator{\Cov}{Cov}         % covariance
\DeclareMathOperator{\Var}{Var}         % variance
\DeclareMathOperator{\Corr}{Corr}       % correlation

% functions and operators
\DeclareMathOperator{\signum}{sign}     % signum/sign of a scalar
\DeclareMathOperator{\dom}{dom}         % domain
\DeclareMathOperator{\epi}{epi}         % epigraph
\DeclareMathOperator{\Ker}{null}        % nullspace/kernel
\DeclareMathOperator{\nullspace}{null}  % nullpsace
\DeclareMathOperator{\range}{range}     % range
\DeclareMathOperator{\Image}{Im}        % image

% topology
\DeclareMathOperator{\interior}{int}    % interior
\DeclareMathOperator{\ri}{rint}         % relative interior
\DeclareMathOperator{\rint}{rint}       % relative interior
\DeclareMathOperator{\bdry}{bdry}       % boundary
\DeclareMathOperator{\cl}{cl}           % closure

% vectors, matrices
\DeclareMathOperator{\linspan}{span}
\DeclareMathOperator{\linspace}{linspace}
\DeclareMathOperator{\cone}{cone}

\DeclareMathOperator{\tr}{tr}           % trace
\DeclareMathOperator{\rank}{rank}       % rank
\DeclareMathOperator{\conv}{conv}       % convex hull
\DeclareMathOperator{\Diag}{Diag}       % Diag(v) = diagonal matrix with v_i on the diagonal
\DeclareMathOperator{\diag}{diag}       % diag(D) = the diagonal vector of matrix D
\DeclareMathOperator{\Arg}{Arg}         % Argument
\DeclareMathOperator*{\argmin}{arg\,min}
\DeclareMathOperator*{\argmax}{arg\,max}

%\renewcommand{\qedsymbol}{\ding{114}}




%%%%%%%%

% TODO: Fix here fonts
\newcommand{\Sample}{\mathcal{S}}
%\newcommand{\norm}[1]{\left\lVert#1\right\rVert_2}
\newcommand{\B}{\mathbb{B}}


\usepackage{tcolorbox}
\usepackage{pifont}
\definecolor{mydarkgreen}{RGB}{39,130,67}
\definecolor{mydarkred}{RGB}{192,47,25}
\definecolor{mydarkorange}{RGB}{39,130,67}
\newcommand{\green}{\color{mydarkgreen}}
\newcommand{\red}{\color{mydarkred}}
\newcommand{\cmark}{\green\ding{51}}%
\newcommand{\xmark}{\red\ding{55}}%
\newcommand{\orange}{\color{mydarkorange}}

\newcommand{\algname}[1]{{\green \sf #1}}
\newcommand{\algnamesmall}[1]{{\green\small \sf #1}}
\newcommand{\algnamelarge}[1]{{\green\large \sf #1}} % for subsections
\newcommand{\algnameLarge}[1]{{\green\Large \sf #1}} % for sections

\newcommand{\tablescriptsize}[1]{{\scriptsize #1}}
\newcommand{\tablesmall}[1]{{\small #1}}



%%%%%%%%

\newtheorem{assumption}{Assumption}
\newtheorem{lemma}{Lemma}
\newtheorem{algorithms}{Algorithm}
\newtheorem{theorem}{Theorem}
\newtheorem{proposition}{Proposition}
\newtheorem{example}{Example}
\newtheorem{corollary}{Corollary}
\newtheorem{fact}{Fact}

\theoremstyle{plain}
\newtheorem{prop}[theorem]{Proposition}
\newtheorem{cor}[theorem]{Corollary}
\newtheorem{lem}[theorem]{Lemma}
\newtheorem{claim}[theorem]{Claim}
\newtheorem{remark}[theorem]{Remark}

\theoremstyle{definition}
\newtheorem{exercise}[theorem]{Exercise}
\newtheorem{rem}[theorem]{Remark}
\newtheorem{que}[theorem]{Question}
\newtheorem{definition}[theorem]{Definition}
\newtheorem{problem}{Problem}

% Hack that enables labeling of lines in algorithms
\newcommand{\alglinelabel}{%
  \addtocounter{ALC@line}{-1}% Reduce line counter by 1
  \refstepcounter{ALC@line}% Increment line counter with reference capability
  \label% Regular \label
}

% Code
\usepackage{listings}
\usepackage{xcolor}

\definecolor{codegreen}{rgb}{0,0.6,0}
\definecolor{codegray}{rgb}{0.5,0.5,0.5}
\definecolor{codepurple}{rgb}{0.58,0,0.82}
\definecolor{backcolour}{rgb}{0.95,0.95,0.92}

\lstdefinestyle{mystyle}{
    backgroundcolor=\color{backcolour},   
    commentstyle=\color{codegreen},
    keywordstyle=\color{magenta},
    numberstyle=\tiny\color{codegray},
    stringstyle=\color{codepurple},
    basicstyle=\ttfamily\footnotesize,
    breakatwhitespace=false,         
    breaklines=true,                 
    captionpos=b,                    
    keepspaces=true,                 
    numbers=left,                    
    numbersep=5pt,                  
    showspaces=false,                
    showstringspaces=false,
    showtabs=false,                  
    tabsize=2
}

\lstset{style=mystyle}
\newcommand{\cI}{{\cal I}}


\usepackage{tikz}
\usetikzlibrary{tikzmark,decorations.pathreplacing}

\hypersetup{
  colorlinks   = true, %Colours links instead of ugly boxes
  urlcolor     = blue, %Colour for external hyperlinks
  linkcolor    = {red!75!black}, %Colour of internal links
  citecolor   = {blue!50!black} %Colour of citations
}

\allowdisplaybreaks

\usepackage{cleveref}
\Crefname{assumption}{Assumption}{Assumptions}  % capitalized references
\Crefname{remark}{Remark}{Remarks}  % capitalized references

\title{Conclusions}
\author{Hanmin Li}

\begin{document}

\maketitle

\tableofcontents

\section{Augmented Lagrangian method}
This note serves as an auxiliary reference to \cite[Chapter 15.1]{beck2017first}.

\begin{tcolorbox}[colback=blue!10, colframe=black, boxrule=0.5pt, width=\textwidth]
    In one sentence, the Augmented Lagrangian method is proximal point method on the dual objective. 
    Notice that in this case, we have strong duality.
\end{tcolorbox}

Consider the problem of 
\begin{align}
    \label{eq:objective}
    H_{opt} = \min \cbrac{H\rbrac{\mx, \mz} = h_1\rbrac{\mx} + h_2\rbrac{\mz}: \mA \mx + \mB \mz = \mc},
\end{align}
where $\mA \in \R^{m \times n}$, $\mB \in \R^{m \times p}$ and $\mc \in \R^m$.
We assume for now that $h_1, h_2$ are proper, closed and convex.

Constructing Lagrangian, we get 
\begin{align*}
    L\rbrac{\mx, \mz; \my} = h_1\rbrac{\mx} + h_2\rbrac{\mz} + \inner{\my}{\mA \mx + \mB \mz - \mc},
\end{align*}
where $\my$ is the Lagrangian multiplier.
Minimizing the Lagrangian, we get the Lagrangian dual function, 
\begin{align*}
    q\rbrac{\my} &= \inf_{\mx \in \R^n, \mz \in \R^p} \cbrac{h_1\rbrac{\mx} + h_2\rbrac{\mz} + \inner{\my}{\mA\mx + \mB\mz - \mc}} \\
    &= -\inner{\my}{\mc} + \inf_{\mx \in \R^n}\cbrac{\inner{\my}{\mA \mx} + h_1\rbrac{\mx}} + \inf_{\mz \in \R^p}\cbrac{\inner{\my}{\mB\mz} + h_2\rbrac{\mz}} \\
    &= -\inner{\my}{\mc} - \sup_{\mx \in \R^d}\cbrac{- \inner{\my}{\mA \mx} - h_1\rbrac{\mx}} - \sup_{\mz \in \R^p}\cbrac{-\inner{\my}{\mB \mz} - h_2\rbrac{\mz}} \\
    &= -\inner{\my}{\mc} - h_1^*\rbrac{-\mA^\top \my} - h_2^*\rbrac{-\mB^\top \my}.
\end{align*}
The dual problem is thus given by 
\begin{align*}
    q_{opt} = \max_{\my \in \R^d} \cbrac{- h_1^*\rbrac{-\mA^\top \my} - h_2^*\rbrac{-\mB^\top \my} - \inner{\my}{\mc}}.
\end{align*}
which is the same as 
\begin{align*}
    \min_{\my \in \R^d} \cbrac{h_1^*\rbrac{-\mA^\top \my} + h_2^*\rbrac{-\mB^\top \my} + \inner{\my}{\mc}}.
\end{align*}
Notice that since $h_1$ and $h_2$ are proper, closed and convex, by \cite[Theorem 4.3 and 4.5]{beck2017first}, we know $h_1^*$ and $h_2^*$ are proper, closed and convex.
Employing \algname{PPM}(Proximal Point Method), we get 
\begin{align*}
    \my_{k+1} = \arg\min_{\mz \in \R^d}\cbrac{h_1^*\rbrac{-\mA^\top\mz} + h_2^\star\rbrac{-\mB^\top \mz} + \inner{\mz}{\mc} + \frac{1}{2\rho}\norm{\mz - \my_k}^2}.
\end{align*}
Using \cite[Theorem 3.40]{beck2017first}, this implies 
\begin{align}
    \label{eq:conv:cond:1}
    & \mzero \in \partial\rbrac{ h_1^\star\rbrac{-\mA^\top \my_{k+1}} } + \partial \rbrac{h_2^*\rbrac{-\mB^\top \my_{k+1}}} + \mc + \frac{1}{\rho}\rbrac{\my_{k+1} - \my_k} \notag \\
    \Leftrightarrow \quad & \mzero \in -\mA \partial h_1^\star\rbrac{-\mA^\top \my_{k+1}} - \mB \partial h_2^*\rbrac{-\mB^\top \my_{k+1}} + \mc + \frac{1}{\rho}\rbrac{\my_{k+1} - \my_k}.
\end{align}
Now the problem comes down to the subdifferential of $h_1^*$ and $h_2^*$, using \cite[Corollary 4.21]{beck2017first}, we have 
\begin{align*}
    & \partial h_1^\star\rbrac{-\mA^\top \my_{k+1}} = \argmax_{\mx \in \R^n}\rbrac{\inner{-\mA^\top \my_{k+1}}{\mx} - h_1\rbrac{\mx}} = \argmin_{\mx \in \R^n}\rbrac{\inner{\mA^\top \my_{k+1}}{\mx} + h_1\rbrac{\mx}} \eqdef \cX_{k+1}, \\
    & \partial h_2^*\rbrac{-\mB^\top \my_{k+1}} = \argmax_{\mz \in \R^p}\rbrac{\inner{-\mB^\top \my_{k+1}}{\mz} - h_2\rbrac{\mz}} = \argmin_{\mz \in \R^p}\rbrac{\inner{\mB^\top \my_{k+1}}{\mz} + h_2\rbrac{\mz}} \eqdef \cZ_{k+1}.
\end{align*}
As a result, $y_{k+1}$ satisfies \cref{eq:conv:cond:1}, if and only if there exists some $\mx_{k+1} \in \cX_{k+1}$, $\mz_{k+1} \in \cZ_{k+1}$, such that 
\begin{align}
    \label{eq:update-rule}
    \my_{k+1} = \my_k + \rho\rbrac{\mA\mx_{k+1} + \mB\mz_{k+1} - \mc}.
\end{align}
To summarize, we plug in the expression of $\my_{k+1}$ into the above condition, and obtain the following equivalent condition,
\begin{eqnarray*}
    \my_{k+1} &=& \my_k + \rho\rbrac{\mA\mx_{k+1} + \mB\mz_{k+1} - \mc} \\
    \mx_{k+1} &\in& \argmin_{\mx \in \R^n}\rbrac{\inner{\mA^\top \rbrac{\my_k + \rho\rbrac{\mA\mx_{k+1} + \mB\mz_{k+1} - \mc}}}{\mx} + h_1\rbrac{\mx}} \\
    \mz_{k+1} &\in& \argmin_{\mz \in \R^p}\rbrac{\inner{\mB^\top \rbrac{\my_k + \rho\rbrac{\mA\mx_{k+1} + \mB\mz_{k+1} - \mc}}}{\mz} + h_2\rbrac{\mz}}.
\end{eqnarray*}
Notice that we have assumed that $h_1\rbrac{x}$ and $h_2\rbrac{z}$ are proper, closed and convex, by Fermat's optimality condition, we have 
\begin{eqnarray*}
    \my_{k+1} &=& \my_k + \rho\rbrac{\mA\mx_{k+1} + \mB\mz_{k+1} - \mc} \\
    \mzero &\in& \mA^\top \rbrac{\my_k + \rho\rbrac{\mA\mx_{k+1} + \mB\mz_{k+1} - \mc}} + \partial h_1\rbrac{\mx_{k+1}} \\
    \mzero &\in& \mB^\top \rbrac{\my_k + \rho\rbrac{\mA\mx_{k+1} + \mB\mz_{k+1} - \mc}} + \partial h_2\rbrac{\mz_{k+1}},
\end{eqnarray*}
which is equivalent to the update rule (\ref{eq:update-rule}).
Notice that $h_1$ and $h_2$ are separable, independently relying on $\mx$ and $\mz$, and the condition reminds us the proximal operator since $x_{k+1}$ and $z_{k+1}$ appears on both sides, which leads to the following condition:

The pair $\rbrac{\mx_{k+1}, \mz_{k+1}}$ is the coordinate-wise minimum of the function via \cite[Lemma 14.7]{beck2017first}, 
\begin{align*}
    \tilde{H}\rbrac{\mx, \mz} := h_1\rbrac{\mx} + h_2\rbrac{\mz} + \frac{\rho}{2}\norm{\frac{1}{\rho}\cdot\my_k + \mA\mx + \mB\mz - \mc}^2.
\end{align*}
As a result we end up in the following Augmented Lagrangian Method (\algname{ALM}).
\vspace{0.3cm}
\begin{tcolorbox}[colback=gray!10, colframe=black, boxrule=0.5pt, width=\textwidth]
    \textbf{The Augmented Lagrangian Method:}
    \\
    
    \textbf{Initialization:} $\mathbf{\my}^0 \in \mathbb{R}^m, \, \rho > 0$.\\
    \textbf{General step:} for any $k = 0, 1, 2, \ldots$ execute the following steps: (primal update) and (dual update):
    \begin{eqnarray*}
        \rbrac{\mx^{k+1}, \mz^{k+1}} &\in& \argmin_{\mx \in \mathbb{R}^n, \mz \in \mathbb{R}^p} \left\{ h_1(\mx) + h_2(\mz) + \frac{\rho}{2} \left\lVert \mA \mx + \mB \mz - \mc + \frac{1}{\rho} \my^k \right\rVert^2
        \right\} \\ 
        \my^{k+1} &=& \my^k + \rho \rbrac{\mA \mathbf{\mx}^{k+1} + \mB \mathbf{\mz}^{k+1} - \mathbf{\mc}}.
    \end{eqnarray*}
\end{tcolorbox}
We therefore define the augmented Lagrangian for objective \ref{eq:objective} as 
\begin{align*}
    L_\rho\rbrac{\mx, \mz; \my} := h_1\rbrac{\mx} + h_2\rbrac{\mz} + \inner{\my}{\mA\mx + \mB\my - \mc} + \frac{\rho}{2}\norm{\mA\mx + \mB\my - \mc}^2.
\end{align*}
$L_0 = L$ is the Lagrangian function. 
The primal step can be written as 
\begin{align*}
    \rbrac{\mx_{k+1}, \mz_{k+1}} \in \argmin_{x \in \R^n, z\in \R^p} L_\rho\rbrac{\mx, \mz, y_k}.
\end{align*}
This algorithm is not in general implementable, since the primal step is complicated.
$\mx$ and $\mz$ ara coupled together.


\section{Alternating direction method of multiplier}
We therefore consider relax the exact minimization in the primal step by one iteration of the alternating minimization method, where we first do minimization with respect to $\mx$, then $\mz$.


\bibliographystyle{plainnat} %abbrv
\bibliography{bibliography}
\end{document}
