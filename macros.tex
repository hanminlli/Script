
%\usepackage{charter}
%\usepackage{calc}

\usepackage{multirow}
% \usepackage[cp1250]{inputenc}
% \usepackage[T1]{fontenc}
% \usepackage{calligra}
\usepackage{layout}

% AMS
\usepackage{amsmath}
\usepackage{amsthm}
\usepackage{amssymb}
\usepackage{amsfonts}
\usepackage{mathtools}
\usepackage{siunitx}

\usepackage{enumitem}

\usepackage{bbm}

\usepackage{url}
\usepackage{color}
\usepackage{graphicx}
\usepackage{verbatim}
\usepackage{subcaption}

\usepackage{natbib}

\usepackage{apptools}
\usepackage{thmtools}
\usepackage{algorithm, algpseudocode}

\usepackage[nice]{nicefrac}

% paper specific macros
\newcommand{\g}{{\color{red}g}}
\newcommand{\h}{{\color{blue}h}}

% general macros

\newcommand{\norm}[1]{\left\| #1 \right\|}
\newcommand{\bnorm}[1]{\left\| #1 \right\|_{\mB}}
\newcommand{\sqnorm}[1]{\left\| #1 \right\|^2}
\newcommand{\lin}[1]{\left\langle #1\right\rangle} % inner product
\newcommand{\inp}[2]{\left\langle#1,#2\right\rangle} % inner product
\newcommand{\binp}[2]{\left\langle#1,#2\right\rangle_{\mB}}

\newcommand{\parens}[1]{\left( #1 \right)}
\newcommand{\brac}[1]{\left\{ #1 \right\}}


% caligraphic
\newcommand{\cA}{\mathcal{A}}
\newcommand{\cB}{\mathcal{B}}
\newcommand{\cC}{\mathcal{C}}
\newcommand{\cD}{\mathcal{D}}
\newcommand{\cF}{\mathcal{F}}
\newcommand{\cH}{\mathcal{H}}
\newcommand{\cL}{\mathcal{L}}
\newcommand{\cK}{\mathcal{K}}
\newcommand{\cJ}{\mathcal{J}}
\newcommand{\cN}{\mathcal{N}}
\newcommand{\cO}{\mathcal{O}}
\newcommand{\cQ}{\mathcal{Q}}
\newcommand{\cS}{\mathcal{S}}
\newcommand{\cT}{\mathcal{T}}
\newcommand{\cU}{\mathcal{U}}
\newcommand{\cW}{\mathcal{W}}
\newcommand{\cX}{\mathcal{X}}
\newcommand{\cY}{\mathcal{Y}}
\newcommand{\cZ}{\mathcal{Z}}

% bold matrices
\newcommand{\mA}{\mathbf{A}}
\newcommand{\mB}{\mathbf{B}}
\newcommand{\mC}{\mathbf{C}}
\newcommand{\mM}{\mathbf{M}}
\newcommand{\mN}{\mathbf{N}}
\newcommand{\mI}{\mathbf{I}}
\newcommand{\mJ}{\mathbf{J}}
\newcommand{\mL}{\mathbf{L}}
\newcommand{\mD}{\mathbf{D}}
\newcommand{\mO}{\mathbf{O}}
\newcommand{\mH}{\mathbf{H}}
\newcommand{\mP}{\mathbf{P}}
\newcommand{\mS}{\mathbf{S}}
\newcommand{\mZ}{\mathbf{Z}}
\newcommand{\mU}{\mathbf{U}}
\newcommand{\mQ}{\mathbf{Q}}
\newcommand{\mV}{\mathbf{V}}
\newcommand{\mX}{\mathbf{X}}
\newcommand{\mW}{\mathbf{W}}

% bold vectors
\newcommand{\ma}{\mathbf{a}}
\newcommand{\mb}{\mathbf{b}}
\newcommand{\mc}{\mathbf{c}}
\newcommand{\md}{\mathbf{d}}
\newcommand{\mx}{\mathbf{x}}
\newcommand{\my}{\mathbf{y}}
\newcommand{\mz}{\mathbf{z}}
\newcommand{\mw}{\mathbf{w}}
\newcommand{\mzero}{\mathbf{0}}


\newcommand{\proj}{\textnormal{Proj}^{\perp}}
\newcommand{\trbar}{\overline{\textnormal{Tr}}}

\newcommand{\prox}{\textnormal{prox}}
\newcommand{\teqp}{t_\textnormal{prox}}
\newcommand{\Teq}{t^*}
\newcommand{\tprox}{t^{\prox}}

\newcommand{\flr}[1]{\left\lfloor #1\right\rfloor} 
\newcommand{\ceil}[1]{\left\lceil #1\right\rceil} 

\newcommand{\hf}{\hat{f}}
\newcommand{\bprox}{\textnormal{brox}_f^t}

% Some duplicate notations on prox and the envelope function
\newcommand{\rbrac}[1]{\left(#1\right)}
\newcommand{\sbrac}[1]{\left[#1\right]}
\newcommand{\cbrac}[1]{\left\{#1\right\}}
\newcommand{\BProx}{\textnormal{brox}}
\newcommand{\BProxSub}[3]{\textnormal{{brox}}^{#1}_{#2}\left(#3\right)}
\newcommand{\ProxSub}[2]{\textnormal{{prox}_{#1}}\left(#2\right)}
\newcommand{\MoreauSub}[3]{{M}^{#1}_{#2}\rbrac{#3}}
\newcommand{\BMoreauSub}[3]{{N}^{#1}_{#2}\rbrac{#3}}
\newcommand{\BMoreau}{{N}}
\newcommand{\inner}[2]{\left\langle #1, #2 \right\rangle}
\newcommand{\dist}[2]{\textnormal{dist}\rbrac{#1, #2}}
\newcommand{\breg}[3]{D_{#1}\rbrac{#2, #3}}
\newcommand{\cmin}[1]{c^{\min}_t\rbrac{x_{#1}}}
\newcommand{\cmax}[1]{c^{\max}_t\rbrac{x_{#1}}}
\newcommand{\BpproxSub}[3]{\textnormal{{bprox}}^{#1}_{#2}\left(#3\right)}
\newcommand{\Bpprox}{\textnormal{bprox}}
\newcommand{\Sprox}{\textnormal{{Sprox}}}
\newcommand{\SproxSub}[4]{\textnormal{{Sprox}}^{#1}_{#2, #3}\rbrac{#4}}


\newcommand{\supp}{\textnormal{supp}}
\newcommand{\prog}{\textnormal{prog}}

\newcommand{\grad}{\nabla}
\newcommand{\sgrad}{\widehat{\nabla}}

% strange stuff
\newcommand{\hotidea}{{\color{red}\bf HOT IDEA: }}
\newcommand{\done}{{\color{blue}\bf DONE: }}
\newcommand{\del}[1]{}
\let\la=\langle
\let\ra=\rangle

% basic macros
\newcommand{\R}{\mathbb{R}} % reals
\newcommand{\Z}{\mathbb{Z}}
\newcommand{\N}{\mathbb{N}}
\newcommand{\I}{\mathbb{I}}
\newcommand{\U}{\mathbb{U}}
\newcommand{\PermComp}{\U\mathbb{P}}

\newcommand{\st}{\;:\;} % such that
\newcommand{\eqdef}{\stackrel{\text{def}}{=}}
% \newcommand{\eqdef}{:=} %\vcentcolon


\newcommand{\clip}[1]{{\rm {\color{mydarkred}clip}_{\tau}}\left(#1\right)}


\newcommand{\nnz}[1]{{\color{red}\|#1\|_0}}

% sets
\DeclareMathOperator{\card}{card}         % cardinality of a set
\DeclareMathOperator{\diam}{diam}        % diameter of a set
\DeclareMathOperator{\vol}{vol}               % volume of a set


\newcommand{\Prob}[1]{{\mathbf{Prob}}\left(#1\right)} % probability
\newcommand{\ProbCond}[2]{\mathbf{Prob}\left(\left.#1\right\vert#2\right)}

% statistics
\newcommand{\Exp}[1]{{\mathbb{E}}\left[#1\right]}
\newcommand{\Expxi}[1]{{\mathbb{E}_{\xi}}\left[#1\right]}
\newcommand{\ExpCond}[2]{{\mathbb{E}}\left[\left.#1\right\vert#2\right]}
\newcommand{\ExpSub}[2]{{\mathbb{E}}_{#1}\left[#2\right]}
\newcommand{\ExpSubCond}[3]{{\mathbb{E}}_{#1}\left[#2\vert#3\right]}
\DeclareMathOperator{\Cov}{Cov}         % covariance
\DeclareMathOperator{\Var}{Var}         % variance
\DeclareMathOperator{\Corr}{Corr}       % correlation

% functions and operators
\DeclareMathOperator{\signum}{sign}     % signum/sign of a scalar
\DeclareMathOperator{\dom}{dom}         % domain
\DeclareMathOperator{\epi}{epi}         % epigraph
\DeclareMathOperator{\Ker}{null}        % nullspace/kernel
\DeclareMathOperator{\nullspace}{null}  % nullpsace
\DeclareMathOperator{\range}{range}     % range
\DeclareMathOperator{\Image}{Im}        % image

% topology
\DeclareMathOperator{\interior}{int}    % interior
\DeclareMathOperator{\ri}{rint}         % relative interior
\DeclareMathOperator{\rint}{rint}       % relative interior
\DeclareMathOperator{\bdry}{bdry}       % boundary
\DeclareMathOperator{\cl}{cl}           % closure

% vectors, matrices
\DeclareMathOperator{\linspan}{span}
\DeclareMathOperator{\linspace}{linspace}
\DeclareMathOperator{\cone}{cone}

\DeclareMathOperator{\tr}{tr}           % trace
\DeclareMathOperator{\rank}{rank}       % rank
\DeclareMathOperator{\conv}{conv}       % convex hull
\DeclareMathOperator{\Diag}{Diag}       % Diag(v) = diagonal matrix with v_i on the diagonal
\DeclareMathOperator{\diag}{diag}       % diag(D) = the diagonal vector of matrix D
\DeclareMathOperator{\Arg}{Arg}         % Argument
\DeclareMathOperator*{\argmin}{arg\,min}
\DeclareMathOperator*{\argmax}{arg\,max}

%\renewcommand{\qedsymbol}{\ding{114}}




%%%%%%%%

% TODO: Fix here fonts
\newcommand{\Sample}{\mathcal{S}}
%\newcommand{\norm}[1]{\left\lVert#1\right\rVert_2}
\newcommand{\B}{\mathbb{B}}


\usepackage{tcolorbox}
\usepackage{pifont}
\definecolor{mydarkgreen}{RGB}{39,130,67}
\definecolor{mydarkred}{RGB}{192,47,25}
\definecolor{mydarkorange}{RGB}{39,130,67}
\newcommand{\green}{\color{mydarkgreen}}
\newcommand{\red}{\color{mydarkred}}
\newcommand{\cmark}{\green\ding{51}}%
\newcommand{\xmark}{\red\ding{55}}%
\newcommand{\orange}{\color{mydarkorange}}

\newcommand{\algname}[1]{{\green \sf #1}}
\newcommand{\algnamesmall}[1]{{\green\small \sf #1}}
\newcommand{\algnamelarge}[1]{{\green\large \sf #1}} % for subsections
\newcommand{\algnameLarge}[1]{{\green\Large \sf #1}} % for sections

\newcommand{\tablescriptsize}[1]{{\scriptsize #1}}
\newcommand{\tablesmall}[1]{{\small #1}}



%%%%%%%%

\newtheorem{assumption}{Assumption}
\newtheorem{lemma}{Lemma}
\newtheorem{algorithms}{Algorithm}
\newtheorem{theorem}{Theorem}
\newtheorem{proposition}{Proposition}
\newtheorem{example}{Example}
\newtheorem{corollary}{Corollary}
\newtheorem{fact}{Fact}

\theoremstyle{plain}
\newtheorem{prop}[theorem]{Proposition}
\newtheorem{cor}[theorem]{Corollary}
\newtheorem{lem}[theorem]{Lemma}
\newtheorem{claim}[theorem]{Claim}
\newtheorem{remark}[theorem]{Remark}

\theoremstyle{definition}
\newtheorem{exercise}[theorem]{Exercise}
\newtheorem{rem}[theorem]{Remark}
\newtheorem{que}[theorem]{Question}
\newtheorem{definition}[theorem]{Definition}
\newtheorem{problem}{Problem}

% Hack that enables labeling of lines in algorithms
\newcommand{\alglinelabel}{%
  \addtocounter{ALC@line}{-1}% Reduce line counter by 1
  \refstepcounter{ALC@line}% Increment line counter with reference capability
  \label% Regular \label
}

% Code
\usepackage{listings}
\usepackage{xcolor}

\definecolor{codegreen}{rgb}{0,0.6,0}
\definecolor{codegray}{rgb}{0.5,0.5,0.5}
\definecolor{codepurple}{rgb}{0.58,0,0.82}
\definecolor{backcolour}{rgb}{0.95,0.95,0.92}

\lstdefinestyle{mystyle}{
    backgroundcolor=\color{backcolour},   
    commentstyle=\color{codegreen},
    keywordstyle=\color{magenta},
    numberstyle=\tiny\color{codegray},
    stringstyle=\color{codepurple},
    basicstyle=\ttfamily\footnotesize,
    breakatwhitespace=false,         
    breaklines=true,                 
    captionpos=b,                    
    keepspaces=true,                 
    numbers=left,                    
    numbersep=5pt,                  
    showspaces=false,                
    showstringspaces=false,
    showtabs=false,                  
    tabsize=2
}

\lstset{style=mystyle}